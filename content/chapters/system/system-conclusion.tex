\section{Conclusion}
In this section we propose two adaptive SPSs, \rSPS{} and \pSPS{}, under a reactive and predictive approach respectively. These adaptive SPSs dynamically modify the number of operator replicas based traffic fluctuations. Both proposals are implemented on top of \textit{Storm}.

Moreover, two other features were included in the two adaptive SPSs. The first consists of a \textit{pool of replicas} aiming at avoiding the restart the SPS when the number of active replicas is modified. The second consists of a grouping strategy, which denoted \textit{Load-Aware} grouping. The latter determines which replica will process the event according to the load of the operator's active replicas, the latter the load balance among the replicas.

The first adaptive presented SPS is \rSPS{}, which is based on a reactive approach. Its aim is to adapt resources according to the analysis of traffic peaks in short periods of time. For this propose, the multi-metric $\delta_i(t)$, called \textit{State} metric is used to define the state of a given operator: overloaded, stable or underloaded. Each state indicates the behaviour of each operator, so that the Planning Algorithm can determine if any modification is necessary. The \textit{State} metric is defined by three weighted metrics: $U_i(t)$ that determines the percentage utilisation of $O_i$ during interval $t$; $E_i(t)$ that determines the processing degradation of the $O_i$ operator during interval $t$; $Q_i(t)$ that determines the impact of queue size on the $O_i$ operator during interval $t$.

The second adaptive presented SPS is \pSPS{}, which is based on a predictive approach. It aims to find patterns in the traffic to predict possible overloads or underloads in the SPS. To this end, the number of replications needed by operator $O_i$ during the next time interval $t+1$, denoted $r_i(t+1)$, is predicted. This value is defined by the average execution time of an event by operator $O_i$ and the predicted number of events received by operator $O_i$ during the next time interval $t+1$. To predict the number of events received by operator $O_i$, the predicted number of events sent by the input data and the number of queued events are considered. Furthermore, the prediction also considers the dependency among operators, in order to mitigate a possible cascade effect due to the change of the number of replicas. For the prediction of events sent by the input data, the use of a predictive model based on mathematical model or artificial intelligence model is proposed.